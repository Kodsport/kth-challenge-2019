\problemname{Assassins}

\illustration{0.49}{chess}{\href{https://www.flickr.com/photos/shyald/414324015}{Picture} by Cristian V.~from flickr, cc by-nd}%%
\noindent
In the cut-throat world of assassins for hire, the rivalry is
ruthless and everyone is fighting to get an edge.  To eliminate the
competition, many assassins even go so far as to assassinate other
assassins.  The question is: with several assassins trying to do
each other in, which ones will remain alive and kicking, and which
ones will kick the bucket?

Assassins generally lay careful plans before executing them, including
planning multiple attempts to dispose of the same target, with the
second attempt being a backup in case the first attempt fails, the
third attempt being a secondary backup, and so on.  Using their great
annihilytical skills, assassins can also very accurately determine the
probability that any given assassination attempt will succeed.

Given the list of planned assassination attempts for a group of
assassins, what are the probabilities that each assassin is alive
after all these attempts?  Performing an assassination attempt
requires that the assassin is still alive, so if the assassin is
indisposed due to already having been assassinated, the attempt is
cancelled.

\section*{Input}

The first line of input contains two integers $n$ and $m$, where $n$
($1 \le n \le 15$) is the number of assassins, and $m$ ($0 \le m \le
1000$) is the number of planned assassination attempts.  The assassins
are numbered from $1$ to $n$.

Then follow $m$ lines, each containing two integers $i$, $j$, and a
real number $p$, indicating that assassin $i$ plans to attempt to
assassinate assassin $j$ ($1 \le i, j \le n$, $j \ne i$), and that
this attempt would succeed with probability $p$ ($0 \le p \le 1$, at
most $6$ digits after the decimal point).  The planned attempts are
listed in chronological order from first to last, and no two attempts
happen simultaneously.

\section*{Output}

Output $n$ lines, with the $i$'th containing the probability that
assassin $i$ is alive after all $m$ assassination attempts have taken
place.  You may assume that none of the $n$ assassins will die of any
other cause than being assassinated in one of these $m$ attempts.  The
probababilities should be accurate to an absolute error of at most
$10^{-6}$.
